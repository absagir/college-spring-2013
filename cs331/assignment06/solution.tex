\documentclass{article}

\usepackage{fancyhdr}
\usepackage{lastpage}
\usepackage{extramarks}
\usepackage[usenames,dvipsnames]{color}
\usepackage{amsmath}
\usepackage{amsthm}
\usepackage{amsfonts}

\usepackage{tikz}

\usetikzlibrary{automata,positioning,calc}

\topmargin=-0.45in
\evensidemargin=0in
\oddsidemargin=0in
\textwidth=6.5in
\textheight=9.0in
\headsep=0.25in

\linespread{1.1}

\pagestyle{fancy}
\lhead{\hmwkAuthorName}
\chead{\hmwkClass\ (\hmwkClassInstructor\ \hmwkClassTime): \hmwkTitle}
\rhead{\firstxmark}
\lfoot{\lastxmark}
\cfoot{}
\renewcommand\headrulewidth{0.4pt}
\renewcommand\footrulewidth{0.4pt}

\setlength\parindent{0pt}

\newcommand{\enterProblemHeader}[1]{
    \nobreak\extramarks{#1}{#1 continued on next page\ldots}\nobreak
    \nobreak\extramarks{#1 (continued)}{#1 continued on next page\ldots}\nobreak
}

\newcommand{\exitProblemHeader}[1]{
    \nobreak\extramarks{#1 (continued)}{#1 continued on next page\ldots}\nobreak
    \nobreak\extramarks{#1}{}\nobreak
}

\setcounter{secnumdepth}{0}
\newcounter{homeworkProblemCounter}

\newcommand{\homeworkProblemName}{}
\newenvironment{homeworkProblem}[1][Problem \arabic{homeworkProblemCounter}]{
    \stepcounter{homeworkProblemCounter}
    \renewcommand{\homeworkProblemName}{#1}
    \section{\homeworkProblemName}
    \enterProblemHeader{\homeworkProblemName}
}{
    \exitProblemHeader{\homeworkProblemName}
}

\newcommand{\problemAnswer}[1]{
\noindent\framebox[\columnwidth][c]{\begin{minipage}{0.98\columnwidth}#1\end{minipage}}
}

\newcommand{\homeworkSectionName}{}
\newenvironment{homeworkSection}[1]{
    \renewcommand{\homeworkSectionName}{#1}
    \subsection{\homeworkSectionName}
    \enterProblemHeader{\homeworkProblemName\ [\homeworkSectionName]}
}{
    \enterProblemHeader{\homeworkProblemName}
}

\newcommand{\hmwkTitle}{Homework\ \#6}
\newcommand{\hmwkDueDate}{March 14, 2013 at 11:59pm}
\newcommand{\hmwkClass}{CS331}
\newcommand{\hmwkClassTime}{9:00am}
\newcommand{\hmwkClassInstructor}{Professor Zhang}
\newcommand{\hmwkAuthorName}{Josh Davis}

\title{
    \vspace{2in}
    \textmd{\textbf{\hmwkClass:\ \hmwkTitle}}\\
    \normalsize\vspace{0.1in}\small{Due\ on\ \hmwkDueDate}\\
    \vspace{0.1in}\large{\textit{\hmwkClassInstructor\ \hmwkClassTime}}
    \vspace{3in}
}

\author{\textbf{\hmwkAuthorName}}
\date{}

\begin{document}

\maketitle

\pagebreak

\begin{homeworkProblem}
    Let \(\Sigma = \{0, 1\}\). Consider the following language over \(\Sigma\),
    \[
         L = \{w_1 \# w_2 : w_1 w_2, \in \{0, 1\}^* \mbox{ and } w_1 < w_2 \}
    \]

    Design a Turing machine using pseudocode that recognizes \(L\).
    \\

    \(M = \)`` On input string \(w\):
    \begin{enumerate}
        \item Start on the left side of the word, move all the way to the right of the input, move to stage \(loop_{>}\)
        \item \textbf{\(loop_>\)} Move to the first character in the second word that isn't an x.  If on 0, mark
            with x and move to stage \textbf{\(cmp_{<0}\)}. If on 1, mark with x and move to \textbf{\(cmp_{>1}\)}.
            If on \textvisiblespace, move to \(q_{reject}\). If on \#, move to stage \textbf{check}.
        \item \textbf{\(loop_<\)} If on 0, mark with x and move to stage \textbf{\(cmp_{>0}\)}. If on 1, mark with x
            and move to \textbf{\(cmp_{<1}\)}. If on \textvisiblespace, move to \(q_{accept}\)
        \item \textbf{\(cmp_{<0}\)} Move to the first character in the first word that isn't an x. If 1, mark with x
            and move to \text{\(loop_>\)}. If 0, mark with x and move to \text{\(loop_<\)}. If \textvisiblespace, move
            to \textbf{\(loop_<\)}.
        \item \textbf{\(cmp_{<1}\)} Move to the first character in the first word that isn't an x. If 1, mark with x
            and move to \text{\(loop_<\)}. If 0, mark with x and move to \text{\(loop_<\)}. If \textvisiblespace,
            move to \textbf{\(loop_<\)}.
        \item \textbf{\(cmp_{>0}\)} Move to the first character in the first word that isn't an x. If 1, mark with x
            and move to \text{\(loop_>\)}. If 0, mark with x and move to \text{\(loop_>\)}. If \textvisiblespace, move
            to \textbf{\(loop_>\)}.
        \item \textbf{\(cmp_{>1}\)} Move to the first character in the first word that isn't an x. If 1, mark with x
            and move to \text{\(loop_>\)}. If 0, mark with x and move to \text{\(loop_<\)}. If \textvisiblespace, move
            to \textbf{\(loop_>\)}.
        \item \textbf{check} Move to first character in the first word that isn't an x. If 1, move to \(q_{reject}\).
        If 0, move to stage \textbf{check}. If \textvisiblespace, move to \(q_{accept}\).''
    \end{enumerate}

    This is valid because we can split up the input into four cases which each have two sub-cases as follows:
    \\
    \textbf{Case One} Input is \(0 w_1 \# 0 w_2\):

    \begin{enumerate}
        \item If \(w_1 < w_2\), then \(w_1 < w_2\)
        \item If \(w_1 > w_2\), then \(w_1 > w_2\)
    \end{enumerate}

    \textbf{Case Two} Input is \(0 w_1 \# 1 w_2\):

    \begin{enumerate}
        \item If \(w_1 < w_2\), then \(w_1 < w_2\)
        \item If \(w_1 > w_2\), then \(w_1 < w_2\)
    \end{enumerate}

    \textbf{Case Three} Input is \(1 w_1 \# 0 w_2\):

    \begin{enumerate}
        \item If \(w_1 < w_2\), then \(w_1 > w_2\)
        \item If \(w_1 > w_2\), then \(w_1 > w_2\)
    \end{enumerate}

    \textbf{Case Four} Input is \(1 w_1 \# 1 w_2\):

    \begin{enumerate}
        \item If \(w_1 < w_2\), then \(w_1 < w_2\)
        \item If \(w_1 > w_2\), then \(w_1 > w_2\)
    \end{enumerate}

    Thus when the word ends, we just accept if we still satisfy \(w_1 < w_2\).
    \\

    \textbf{Note} When a given word runs out of characters, we just treat it if it had
    0's appended to the beginning such that the length of the words are equal.
\end{homeworkProblem}

\pagebreak

\begin{homeworkProblem}
    Prove that 2-dimensional Turing machines are no more powerful than standard Turing
    machines.

    \begin{proof}
        We will use a proof by construction to construct a new 2D TM, \(N\) such that
        it can be simulated by a normal TM, \(M\). Let \(N\) and \(M\) be defined as follows:
        \[
            N = (Q, \Sigma, \Gamma, \delta, q, q_{accept}, q_{reject})
            M = (Q_M, \Sigma_M, \Gamma_M, \delta_M, q_{M}, m_{accept}, m_{reject})
        \]

        We will use a counting argument to show that the tape of \(N\) is identical to
        the tape of \(M\) in size.
        \\

        Much like the way we show that the number of rational numbers is countably infinite
        by making a 1-to-1 correspondence to \(\mathbb{N}\). We can do the exact same thing as
        this.
        \\

        Achieving this will show that every square on 2D tape corresponds to tape on the normal TM.

        \begin{table}[ht]
            \centering
            \begin{tabular}{c || c | c | c | c | c}
                 & 0 & 1 & 2 & 3 & 4 \\
                \hline
                0 & (0, 0) & (0, 1) & (0, 2) & (0, 3) & (0, 4) \\
                1 & (1, 0) & (1, 1) & (1, 2) & (1, 3) & (1, 4) \\
                2 & (2, 0) & (2, 1) & (2, 2) & (2, 3) & (2, 4) \\
                3 & (3, 0) & (3, 1) & (3, 2) & (3, 3) & (3, 4) \\
                4 & (4, 0) & (4, 1) & (4, 2) & (4, 3) & (4, 4) \\
            \end{tabular}
        \end{table}

        By moving in the order \((0, 0) \rightarrow (0, 1) \rightarrow (1, 0)
        \rightarrow (0, 2) \rightarrow \hdots\), the number of squares can be assigned
        with a 1-to-1 correspondence to \(\mathbb{N}\). Thus this shows that a normal TM tape
        of \(M\) can simulate a 2D tape of \(N\)
        \\

        Given this construction, it can easily be seen that now moving up,
        down, right, and left are possible by just moving to the corresponding
        square in the sequence.
        \\

        Considering that is the hardest part, the rest of \(N\) can be defined as follows:

        \begin{enumerate}
            \item \(Q = Q_M\) The states are the same
            \item \(\Sigma = \Sigma_M\) The alphabet is the same
            \item \(\Gamma = \Gamma_M \) The stack alphabet is the same
            \item \(q = q_M\) The start states are the same
            \item \(q_{accept} = m_{accept}\) The accept states are the same
            \item \(q_{reject} = m_{reject}\) The reject states are the same
        \end{enumerate}

        Thus we have shown that a 2D TM, \(N\) is no more powerful than a normal
        TM, \(M\) by constructing \(N\) out of \(M\). Our proof is complete.
    \end{proof}
\end{homeworkProblem}

\pagebreak

\begin{homeworkProblem}
    Prove that the class of Turing-recognizable languages are closed under certain operations.
    \\

    \textbf{Part One} Union: \(L = L_1 \cup L_2\)
    \begin{proof}
        Given two Turing machines, \(M_1\) and \(M_2\) that recognize the languages \(L_1\)
        and \(L_2\), respectively, we will construct a new Turing machine, \(M\), that recognizes
        the union of the two languages.
        \\

        Let \(M_1\) and \(M_2\) be defined as follows:
        \[
            \begin{split}
                M_1 = (Q_1, \Sigma_1, \Gamma_1, \delta_1, q_{M10}, q_{M1accept}, q_{M1reject})
                \\
                M_2 = (Q_2, \Sigma_2, \Gamma_2, \delta_2, q_{M20}, q_{M2accept}, q_{M2reject})
            \end{split}
        \]

        We can then construct \(M\) as follows:

        \begin{enumerate}
            \item \(Q = Q_1 \cup Q_2\), the state machine
            \item \(\Sigma = \Sigma_1 \cup \Sigma_2\), the alphabet
            \item \(\Gamma = \Gamma_1 \cup \Gamma_2 \), the stack alphabet
            \item \(q_{accept}\) A new accept state for when both machines accept.
            \item \(q_{0}\) A new start state.
            \item \(M = \)`` On input string \(w\):
                \begin{enumerate}
                    \item Run the input \(w\) over the two TMs in parallel.
                    \item If either \(M_1\) or \(M_2\) accept the input, move to the accept state
                        if both machines reject it, move to the reject state.
                \end{enumerate}
        \end{enumerate}

        This works because we can run the TMs in parallel and accept if either of them accepts. This
        shows that the language will accept words that are in either the language of the first TM
        or the secnod one.
    \end{proof}

    \pagebreak

    \textbf{Part Two} Intersection: \(L = L_1 \cap L_2\)
    \begin{proof}
        Given two Turing machines, \(M_1\) and \(M_2\) that recognize the languages \(L_1\)
        and \(L_2\), respectively, we will construct a new Turing machine, \(M\), that recognizes
        the intersection of the two languages.
        \\

        Let \(M_1\) and \(M_2\) be defined as follows:
        \[
            \begin{split}
                M_1 = (Q_1, \Sigma_1, \Gamma_1, \delta_1, q_{M10}, q_{M1accept}, q_{M1reject})
                \\
                M_2 = (Q_2, \Sigma_2, \Gamma_2, \delta_2, q_{M20}, q_{M2accept}, q_{M2reject})
            \end{split}
        \]

        We can then construct \(M\) as follows:

        \begin{enumerate}
            \item \(Q = Q_1 \cup Q_2\), the state machine
            \item \(\Sigma = \Sigma_1 \cup \Sigma_2\), the alphabet
            \item \(\Gamma = \Gamma_1 \cup \Gamma_2 \), the stack alphabet
            \item \(q_{accept}\) A new accept state for when both machines accept.
            \item \(q_{0}\) A new start state.
            \item \(M = \)`` On input string \(w\):
                \begin{enumerate}
                    \item Run the input \(w\) over the two TMs in parallel.
                    \item If both \(M_1\) and \(M_2\) accept the input, move to the accept state
                        if either machine rejects it, move to the reject state.
                \end{enumerate}
        \end{enumerate}

        This works because we can just run the two TMs in parallel and only accept when
        they both accept. Thus a word will only appear in our new language when both TMs accept it
        and don't reject it.
    \end{proof}
    
    \pagebreak

    \textbf{Part Three} Concatentation: \(L = L_1 * L_2\)
    \begin{proof}
        Given two Turing machines, \(M_1\) and \(M_2\) that recognize the languages \(L_1\)
        and \(L_2\), respectively, we will construct a new Turing machine, \(M\), that recognizes
        the concatentation of the two languages.
        \\

        Let \(M_1\) and \(M_2\) be defined as follows:
        \[
            \begin{split}
                M_1 = (Q_1, \Sigma_1, \Gamma_1, \delta_1, q_{M10}, q_{M1accept}, q_{M1reject})
                \\
                M_2 = (Q_2, \Sigma_2, \Gamma_2, \delta_2, q_{M20}, q_{M2accept}, q_{M2reject})
            \end{split}
        \]

        We can then construct \(M\) as follows:

        \begin{enumerate}
            \item \(Q = Q_1 \cup Q_2\), the state machine
            \item \(\Sigma = \Sigma_1 \cup \Sigma_2\), the alphabet
            \item \(\Gamma = \Gamma_1 \cup \Gamma_2 \), the stack alphabet
            \item \(q_{accept}\) A new accept state for when both machines accept.
            \item \(q_{0}\) A new start state.
            \item \(M = \)`` On input string \(w\):
                \begin{enumerate}
                    \item Split the input \(w\) into two parts, \(w = xy\). We
                        can do this nondeterministically.
                    \item Then run \(M_1\) over the first part, \(x\), if it
                        accepts it go to the next stage, if it rejects it, move
                        to the reject state
                    \item Next, run \(M_2\) over the second part, \(y\). If it
                        accepts it, move to the accept state, if it rejects it,
                        move to the reject state.
                \end{enumerate}
        \end{enumerate}

        This works because we can use the power of Nondeterministic TMs to split the string and run each
        TM sequentially. This is valid because it is shown in the book that NTMs are no more powerful than
        normal TMs. In splitting the string, we can assure that every possibility will be checked and thus
        it will show that the first language is the concatenation of the second one.
    \end{proof}
\end{homeworkProblem}

\pagebreak

\begin{homeworkProblem}
    Prove the following languages aren't Turing decidable.
    \\

    \textbf{Part One} A TM, \(N\) will write "V" somewhere on the tape, \(L_B\).

    \begin{proof}
        We assume that \(N\) is decidable and obtain a contradiction. Suppose
        \(N\) is a decider for \(L_B\). Then \(N\) is a TM and it will accept
        any \(\left<M\right>\) and either accept or reject it.
        \\

        If we order the language, \(L_B\), we can sort it by when the first occurence of "V"
        happens on the tape and label it \(M_i\) where \(i\) is the index of
        the occurrence. This will give us a table as below where x indicates
        any other character:

        \begin{table}[ht]
            \centering
            \begin{tabular}{c || c | c | c | c | c}
                 & 1 & 2 & 3 & 4 & \(\hdots\) \\
                \hline
                \(\left<M_1\right>\) & V & x & x & x & x \\
                \(\left<M_2\right>\) & x & V & x & x & x \\
                \(\left<M_3\right>\) & x & x & V & x & x \\
                \(\left<M_4\right>\) & x & x & x & V & x \\
                \(\left<M_5\right>\) & x & x & x & x & V \\
            \end{tabular}
        \end{table}

        To prove that this is a contradiction, all \(M_i\) should be decidable by \(N\).
        However, if we use the diagonalization method, we can see that if we select a diagonal
        through the table and change it such that the input becomes something else, it would make
        a new input, \(y = xxxxxx\hdots\). This new input must exist somewhere along the rows but
        as we know, it won't because at the point where this new input crosses the diagonal, it won't
        be equal to the value located there.
        \\

        Thus we have found a contradiction and \(L_B\) is not decidable.
    \end{proof}

    \textbf{Part Two} A TM, \(N\) halts on all words except one.

    \begin{proof}
        We assume that \(N\) is decidable and obtain a contradiction. Suppose
        \(N\) is a decider for \(L_U\). Then \(N\) is a TM and it will accept
        any \(\left<M\right>\) and either accept or reject it.
        \\

        If we order the language, \(L_U\), we can sort it by when the word that
        the machine happens on the tape and label it \(M_i\) where \(i\) is the
        index of the occurrence. This will give us a table as below where x
        indicates any other character:

        \begin{table}[ht]
            \centering
            \begin{tabular}{c || c | c | c | c | c}
                 & 1 & 2 & 3 & 4 & \(\hdots\) \\
                \hline
                \(\left<M_1\right>\) & doesn't halt & halts & halts & halts & halts \\
                \(\left<M_2\right>\) & halts & doesn't halt & halts & halts & halts \\
                \(\left<M_3\right>\) & halts & halts & doesn't halt & halts & halts \\
                \(\left<M_4\right>\) & halts & halts & halts & doesn't halt & halts \\
                \(\left<M_5\right>\) & halts & halts & halts & halts & doesn't halt \\
            \end{tabular}
        \end{table}

        To prove that this is a contradiction, all \(M_i\) should be decidable by \(N\).
        However, if we use the diagonalization method, we can see that if we select a diagonal
        through the table and change it such that the input becomes the opposite, it would make
        a new input, \(y\), that doesn't halt for all input. This new input
        must exist somewhere along the rows but as we know, it won't because at
        the point where this new input crosses the diagonal, it won't
        be equal to the value located there.
        \\

        Thus we have found a contradictino and \(L_U\) is not decidable.
    \end{proof}
\end{homeworkProblem}

\pagebreak

\begin{homeworkProblem}
    Prove or disprove that the language, \(L_s\) is Turing decidable.
    \\

    \begin{proof}
        We will show that the language \(L_s\) is in fact Turing decidable.
        \\

        To show that this is true, we will use the idea of Configuration History
        to prove it.
        \\

        Given a TM, \(M = (Q, \Sigma, \Gamma, \delta, q, q_{accept}, q_{reject})\), we
        can keep track of the configurations. This can be represented as follows:
        \[
            c_0 \rightarrow c_1 \rightarrow \hdots \rightarrow c_{n-1} \rightarrow c_n
        \]

        \(c_0\) will be the starting state while \(c_n\) will be the accepting state (if it halts).
        \\

        Since we have this configuration, we can deterministically represent
        all the possible states that the configurations can be in. The total
        number of possibilities is \(n^k(k)(m)\) where \(k =
        \left|\Gamma\right|\), \(m = \left|Q\right|\).
        \\

        Thus each configuration step we move along, we can compare it to the last. If the configuration
        is in the same state, it must be in a loop and we reject. If it isn't in the same state, continue.
        \\

        Doing this we can turn the language, \(L_s\) into a decidable language by bounding the number of
        movements the head can make.
    \end{proof}
\end{homeworkProblem}

\end{document}
