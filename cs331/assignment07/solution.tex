\documentclass{article}

\usepackage{fancyhdr}
\usepackage{lastpage}
\usepackage{extramarks}
\usepackage[usenames,dvipsnames]{color}
\usepackage{amsmath}
\usepackage{amsthm}
\usepackage{amsfonts}

\usepackage{tikz}

\topmargin=-0.45in
\evensidemargin=0in
\oddsidemargin=0in
\textwidth=6.5in
\textheight=9.0in
\headsep=0.25in

\linespread{1.1}

\pagestyle{fancy}
\lhead{\hmwkAuthorName}
\chead{\hmwkClass\ (\hmwkClassInstructor\ \hmwkClassTime): \hmwkTitle}
\rhead{\firstxmark}
\lfoot{\lastxmark}
\cfoot{}
\renewcommand\headrulewidth{0.4pt}
\renewcommand\footrulewidth{0.4pt}

\setlength\parindent{0pt}

\newcommand{\enterProblemHeader}[1]{
    \nobreak\extramarks{#1}{#1 continued on next page\ldots}\nobreak
    \nobreak\extramarks{#1 (continued)}{#1 continued on next page\ldots}\nobreak
}

\newcommand{\exitProblemHeader}[1]{
    \nobreak\extramarks{#1 (continued)}{#1 continued on next page\ldots}\nobreak
    \nobreak\extramarks{#1}{}\nobreak
}

\setcounter{secnumdepth}{0}
\newcounter{homeworkProblemCounter}

\newcommand{\homeworkProblemName}{}
\newenvironment{homeworkProblem}[1][Problem \arabic{homeworkProblemCounter}]{
    \stepcounter{homeworkProblemCounter}
    \renewcommand{\homeworkProblemName}{#1}
    \section{\homeworkProblemName}
    \enterProblemHeader{\homeworkProblemName}
}{
    \exitProblemHeader{\homeworkProblemName}
}

\newcommand{\problemAnswer}[1]{
\noindent\framebox[\columnwidth][c]{\begin{minipage}{0.98\columnwidth}#1\end{minipage}}
}

\newcommand{\homeworkSectionName}{}
\newenvironment{homeworkSection}[1]{
    \renewcommand{\homeworkSectionName}{#1}
    \subsection{\homeworkSectionName}
    \enterProblemHeader{\homeworkProblemName\ [\homeworkSectionName]}
}{
    \enterProblemHeader{\homeworkProblemName}
}

\newcommand{\hmwkTitle}{Homework\ \#7}
\newcommand{\hmwkDueDate}{April 1, 2013 at 11:59pm}
\newcommand{\hmwkClass}{CS331}
\newcommand{\hmwkClassTime}{9:00am}
\newcommand{\hmwkClassInstructor}{Professor Zhang}
\newcommand{\hmwkAuthorName}{Josh Davis}

\title{
    \vspace{2in}
    \textmd{\textbf{\hmwkClass:\ \hmwkTitle}}\\
    \normalsize\vspace{0.1in}\small{Due\ on\ \hmwkDueDate}\\
    \vspace{0.1in}\large{\textit{\hmwkClassInstructor\ \hmwkClassTime}}
    \vspace{3in}
}

\author{\textbf{\hmwkAuthorName}}
\date{}

\begin{document}

\maketitle

\pagebreak

\begin{homeworkProblem}
    Let \(K = \{\left<M\right> : \left<M\right> \notin L(M)\}\). Prove that \(K\) is not Turing
    recognizable.

    \begin{proof}
        To prove that \(K\) is not Turing recognizable, we will prove this by coming up with a contradiction.
        \\

        Assume that \(K\) is Turing recognizable and \(N\) is a Turing machine
        that recognizes it. \(K\) is the set of all encoded Turing machines
        that recognize the language that doesn't include their own encoding.
        \\

        Using this assumption, we can see that if we take the TM that
        recognizes \(K\), \(N\), we can see there are two cases in which this
        property of \(K\) still holds.
        \\

        \textbf{Case One}
        \\

        If \(N \in K\) then that means \(N\) recognizes the language of all
        encoded TMs that don't recognize themselves. This also means that \(N\)
        must now be removed from the language because it cannot contain itself.
        Thus \(N\) cannot be in \(K\).
        \\

        \textbf{Case Two}
        \\

        If \(N \notin K\) then that means \(N\) recognizes the language of all encoded
        TMS that don't recognize themselves. This also means that \(N\), by the definition,
        doesn't recognize itself. Thus it must be added to \(K\).
        \\

        Thus \(N \in K \iff N \notin K\).  This is clearly a contradiction and
        thus we have shown that \(K\) is not Turing recognizable.
    \end{proof}

\end{homeworkProblem}

\pagebreak

\begin{homeworkProblem}
    Prove the following statements:
    \begin{enumerate}
        \item \(L_1 \leq_{m} L_2\) and \(L_2 \leq_{m} L_3\) imply \(L_1 \leq_{m} L_3\).
        \item \(L_1 \leq_{T} L_2\) implies that \(\overline{L_1} \leq_{m} \overline{L_2}\).
    \end{enumerate}

    \textbf{Part One}

    \begin{proof}
        Let there be three TMs that decide the three languages. \(M_1\)
        recognizes \(L_1\), \(M_2\) recognizes \(L_2\), and \(M_3\) recognizes
        \(L_3\).
        \\

        According to many-one reduction, if one language reduces to another
        language, that means there is a computable function such that the
        function reduces the first language to the second.
        \\

        Therefore with \(L_1 \leq_{m} L_2\), there is a computable function,
        \(f_{12}\) where for every \(w\), \(w \in L_1 \iff f_{12}(w) \in L_2\).
        \\

        The same can be said for \(L_2 \leq_{m} L_3\), there is a computable
        function, \(f_{23}\) where for every \(w\), \(w \in L_2 \iff f_{23}(w) \in
        L_3\).
        \\

        Now to prove that \(L_1 \leq_{m} L_2\) and \(L_2 \leq_{m} L_3\) imply
        \(L_1 \leq_{m} L_3\), we will construct a new TM, \(N\), that
        recognizes \(L_1 \leq_{m} L_3\). We will construct \(N\) using the following
        computable function, \(f(w) = f_{23}(f_{12}(w))\):
        \\

        \(N = \)`` On input string \(w\):
        \begin{enumerate}
            \item Run \(M_1\) on \(w\), if it accepts, move on, else reject
            \item Compute \(f(w)\) using our new computable function
            \item Run \(M_3\) on the previously computed value, output whatever \(M_3\) outputs''
        \end{enumerate}

        Thus we have shown that when \(L_1\) reduces to \(L_2\) and \(L_2\) reduces to \(L_3\),
        we can construct a new many-reduction that reduces \(L_1\) to \(L_3\). Thus we have proved
        what we sought to prove and our proof is complete.
    \end{proof}

    \textbf{Part Two}

    \begin{proof}
        Since \(L_1\) is Turing reducible to \(L_2\), that means \(L_1\) is
        deciable relative to \(L_2\) and there is an oracle TM, \(M^1\), that can report
        whether or not any string \(w\) is a member of \(L_1\). Likewise there
        is another oracle TM, \(M^2\), that can report whether or not any
        string \(w\) is a member of \(L_2\).
        \\

        To show that this then implies that \(\overline{L_1} \leq_{m} \overline{L_2}\),
        we can construct a new oracle TM using these existing oracle TMs. Let
        our new oracle TM be \(N\) and defined as follows:
        \\

        \(N = \)`` On input string \(w\):
        \begin{enumerate}
            \item Query \(M^1\) with \(w\), if it answers \textbf{no}, continue, else reject
            \item Next query \(M^2\) with \(w\), if it answers \textbf{no} accept, else reject''
        \end{enumerate}

        This is valid because when we query the oracles, since we want the
        complement of the languages, we just answer the opposite of what the
        oracle tells us. Thus we can create a new oracle from the oracles for
        \(L_1\) and \(L_2\) that can show that \(L_1 \leq_{T} L_2\) implies
        that \(\overline{L_1} \leq_{m} \overline{L_2}\).
        \\

        Thus we have proven what we wanted to and our proof is complete.

    \end{proof}
\end{homeworkProblem}

\pagebreak

\begin{homeworkProblem}
    Prove that \(A_{TM} \not\leq_{m} \overline{A_{TM}}\), where \(A_{TM} =
    \{\left<M, w\right> : M \mbox{ is a TM and } M \mbox{ accepts } w \}\)

    \begin{proof}
        To prove this, we will do a proof by contradiction.
        \\

        Suppose that \(A_{TM} \leq_{m} \overline{A_{TM}}\), or \(A_{TM}\)
        reduces to \(\overline{A_{TM}}\). This means that there is a computable
        function, \(f\) that on every input \(w\), \(w
        \in A_{TM} \iff f(w) \in \overline{A_{TM}}\). Let \(M\) be the TM that
        recognizes \(A_{TM}\).
        \\

        According to the above, we can construct a new TM, \(N\) that
        recognizes \(\overline{A_{TM}}\). Let \(N\) be constructed as follows:

        \(N = \)`` On input string \(w\):
        \begin{enumerate}
            \item Compute \(f(w)\)
            \item Run \(M\) with \(w\), if \(M\) rejects, then accepts and if \(M\) accepts, then reject''
        \end{enumerate}

        As we can see, the only way to that \(A_{TM}\) reduces to
        \(\overline{A_{TM}}\) is if \(A_{TM}\) is decidable. Since we know that
        \(A_{TM}\) is not decidable, we have a contradiction.
        \\

        Since we have a contradiction \(A_{TM} \leq_{m}
        \overline{A_{TM}}\) cannot be true and we have concluded our proof.
    \end{proof}
\end{homeworkProblem}

\begin{homeworkProblem}
    Which of of the following PCP problems has a solution? Justify.

    \Large
    \begin{enumerate}
        \item \(\{\left[\frac{ab}{a}\right], \left[\frac{bb}{ab}\right],
                \left[\frac{aa}{ba}\right], \left[\frac{cc}{bc}\right],
                \left[\frac{aa}{ca}\right], \left[\frac{d}{cd}\right]\}\)
        \item \(\{\left[\frac{ab}{a}\right], \left[\frac{bb}{ab}\right],
                \left[\frac{aa}{ba}\right], \left[\frac{c}{bc}\right],
                \left[\frac{aa}{ca}\right], \left[\frac{d}{cd}\right]\}\)
    \end{enumerate}
    \normalsize

    \textbf{Part One}
    \\

    One possible solution is below:
    \[
        \left[\frac{ab}{a}\right] \left[\frac{cc}{bc}\right] \left[\frac{d}{cd}\right]
    \]

    This is a solution because if we read across the top, we get \(abccd\) and
    if we read across the bottom we get \(abccd\). Thus we have solved this
    given PCB problem.
    \\

    \textbf{Part Two}
    \\

    One possible solution is below:
    \[
        \left[\frac{ab}{a}\right] \left[\frac{aa}{ba}\right] \left[\frac{bb}{ab}\right] \left[\frac{c}{bc}\right]
    \]

    This is a solution because if we read across the top, we get \(abaabbc\) and
    if we read across the bottom we get \(abaabcc\). Thus we have solved this
    given PCB problem.
\end{homeworkProblem}

\pagebreak

\begin{homeworkProblem}
    Does the following PCP problem P have a solution?

    \[
        \{
            \left[\frac{a}{ab}\right],
            \left[\frac{b}{ccc}\right],
            \left[\frac{c}{b}\right],
            \left[\frac{c}{d}\right],
            \left[\frac{dddd}{}\right],
            \left[\frac{ddde}{e}\right],
        \}
    \]

    Yes, it has a solution. One possible solution is below:

    \[
        \left[\frac{a}{ab}\right]
        \left[\frac{b}{ccc}\right]
        \left[\frac{c}{d}\right]
        \left[\frac{c}{d}\right]
        \left[\frac{c}{d}\right]
        \left[\frac{ddde}{e}\right]
    \]

    This is a solution because if we read across the top, we get \(abcccddde\)
    and if we read across the bottom we get \(abcccddde\). Thus we have solved
    this given PCB problem.

    \begin{proof}
        To show that the PCP problem P has a solution, we wil show that P has a solution
        if and only if there exists an \(n\) such that \((3^n \mod 4) = 3\).
        \\

        If \((3^n \mod 4) = 3\), then there must be \(n\) of tile 2. Since
        there are \(3^n\) \(c\)'s, they must be matched on the top with \(c\)'s
        as well. This will either give zero \(d\)'s on the bottom or \(3^n\)
        \(d\)'s as well. Thus the next tile that needs to be used is the 5th
        tile. The 5th tile needs to be used until \(3^n \mod 4 = 3\). Then we
        can add the last tile, making a solution.  \\

        Likewise the opposite is similar, if there is a solution, then the
        number of \(d\)'s must match on the top and the bottom. Since we can
        only add more \(d\)'s to the bottom by matching \(c\)'s, we must add 3
        at a time. Then we need to use the 4th tile until there are only three
        \(d\)'s left. Then we add the last tile and the solution is complete.
        \\

        Since we have proven both ways, we have shown that a solution can exist
        iff there exists an \(n\) such that \((3^n \mod 4) = 3\).

    \end{proof}

\end{homeworkProblem}

\pagebreak

\begin{homeworkProblem}
    Show that \(BB(k)\) is not a computable function.

    \begin{proof}
        We will show that \(BB(k)\) is not a computable function by proof by
        contradiction.
        \\

        Assume that \(BB(k)\) is a computable function. Since \(BB(k)\) is a
        computable function, we can represent this function as a Turing
        machine. Let's name this TM, \(M\).
        \\

        According to the definition of \(BB(k)\), it is able to give us the maximum
        number of steps for a \(k\) state TM. Using this, we can determine if a machine
        halts. Thus we will use \(BB(k)\) to decide \(HALT_{TM}\).
        \\

        We will now construct a TM \(N\) that decides \(HALT_{TM}\).
        \\

        \(N = \)`` On input string \(\langle M, w\rangle\):
        \begin{enumerate}
            \item Decode \(M\) and count the states, let it be \(k\)
            \item Compute \(k\) with the Busy Beaver TM, \(M\), let this be \(n\)
            \item Now execute \(M\) with \(w\), counting each step along the way, let this be \(m\)
            \item If \(M\) ever rejects or accepts, then \(accept\).
            \item If \(m\) ever exceeds \(n\), then \(reject\).
        \end{enumerate}

        This is a contradiction because if we can solve \(BB(k)\), then we can
        solve \(HALT_{TM}\). We know that \(HALT_{TM}\) is undecidable which
        means that \(BB(k)\) must also be undecidable. Thus we have shown that \(BB(k)\) is not
        a computable function and our proof is complete.
    \end{proof}
\end{homeworkProblem}

\end{document}
