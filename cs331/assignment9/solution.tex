\documentclass{article}

\usepackage{fancyhdr}
\usepackage{lastpage}
\usepackage{extramarks}
\usepackage[usenames,dvipsnames]{color}
\usepackage{amsmath}
\usepackage{amsthm}
\usepackage{amsfonts}

\usepackage{tikz}

\topmargin=-0.45in
\evensidemargin=0in
\oddsidemargin=0in
\textwidth=6.5in
\textheight=9.0in
\headsep=0.25in

\linespread{1.1}

\pagestyle{fancy}
\lhead{\hmwkAuthorName}
\chead{\hmwkClass\ (\hmwkClassInstructor\ \hmwkClassTime): \hmwkTitle}
\rhead{\firstxmark}
\lfoot{\lastxmark}
\cfoot{}
\renewcommand\headrulewidth{0.4pt}
\renewcommand\footrulewidth{0.4pt}

\setlength\parindent{0pt}

\newcommand{\enterProblemHeader}[1]{
    \nobreak\extramarks{#1}{#1 continued on next page\ldots}\nobreak
    \nobreak\extramarks{#1 (continued)}{#1 continued on next page\ldots}\nobreak
}

\newcommand{\exitProblemHeader}[1]{
    \nobreak\extramarks{#1 (continued)}{#1 continued on next page\ldots}\nobreak
    \nobreak\extramarks{#1}{}\nobreak
}

\setcounter{secnumdepth}{0}
\newcounter{homeworkProblemCounter}

\newcommand{\homeworkProblemName}{}
\newenvironment{homeworkProblem}[1][Problem \arabic{homeworkProblemCounter}]{
    \stepcounter{homeworkProblemCounter}
    \renewcommand{\homeworkProblemName}{#1}
    \section{\homeworkProblemName}
    \enterProblemHeader{\homeworkProblemName}
}{
    \exitProblemHeader{\homeworkProblemName}
}

\newcommand{\problemAnswer}[1]{
\noindent\framebox[\columnwidth][c]{\begin{minipage}{0.98\columnwidth}#1\end{minipage}}
}

\newcommand{\homeworkSectionName}{}
\newenvironment{homeworkSection}[1]{
    \renewcommand{\homeworkSectionName}{#1}
    \subsection{\homeworkSectionName}
    \enterProblemHeader{\homeworkProblemName\ [\homeworkSectionName]}
}{
    \enterProblemHeader{\homeworkProblemName}
}

\newcommand{\hmwkTitle}{Homework\ \#8}
\newcommand{\hmwkDueDate}{April 11, 2013 at 11:59pm}
\newcommand{\hmwkClass}{CS331}
\newcommand{\hmwkClassTime}{9:00am}
\newcommand{\hmwkClassInstructor}{Professor Zhang}
\newcommand{\hmwkAuthorName}{Josh Davis}

\title{
    \vspace{2in}
    \textmd{\textbf{\hmwkClass:\ \hmwkTitle}}\\
    \normalsize\vspace{0.1in}\small{Due\ on\ \hmwkDueDate}\\
    \vspace{0.1in}\large{\textit{\hmwkClassInstructor\ \hmwkClassTime}}
    \vspace{3in}
}

\author{\textbf{\hmwkAuthorName}}
\date{}

\begin{document}

\maketitle

\pagebreak

\begin{homeworkProblem}
    \textbf{Part A} Prove that \(n! \in O(n^n)\).
    \\

    \begin{proof}
        To prove that \(n! \in O(n^n)\), we will use the definition of big-O
        which states: For two functions, \(f\) and \(g\). Say that \(f(n) =
        O(g(n))\) if positive integers \(c\) and \(n_0\) exist such that for
        every integer \(n \geq n_0\) that \(f(n) \leq cg(n)\).
        \\

        Thus we need to find a \(c\) and \(n_0\) to satisfy the above
        condition.
        \\

        Let \(f(n) = n! \) and \(g(n) = n^n\). Let \(c = 1\) and \(n_0 = 1\), so for
        all \(n \geq n_0\) we have:

        \[
            \begin{split}
                n! &= (n)(n - 1)(n - 2)\hdots(2)(1)
                \\
                &\leq c(n_n)(n_{n-1})(n_{n-2})\hdots(n_{2})(n_{1})
                \\
                &= cn^n
            \end{split}
        \]

        Since we have satisfied the definition of big-O, we can see that \(n!
        \in O(n^n)\) and thus our proof is complete.
    \end{proof}

    \textbf{Part B} Which of the following relations is true and which is false?

    \begin{enumerate}
        \item \(n \in O((\lg n)^3)\)
            \\
            \textbf{True | False}

        \item \((\lg n)^3 \in o(n)\)
            \\
            \textbf{True}
            \[
                \lim_{n \to \infty} \frac{(\lg n)^3}{n} = 0
            \]
        \item \(n^{\lg n} \in O(2^{n \lg n})\)
            \\
            \textbf{True | False}
        \item \(n^4 \in o(100n^4)\)
            \\
            \textbf{False}
            \[
                    \lim_{n \to \infty} \frac{n^4}{100n^4}
                    = \lim_{n \to \infty} \frac{1}{100}
                    = \frac{1}{100}
            \]
        \item \((\lg n)^n \in O(\sqrt{2^n})\)
            \\
            \textbf{True | False}
    \end{enumerate}
\end{homeworkProblem}

\pagebreak

\begin{homeworkProblem}
    Prove that any language in \(P\) is \textbf{polynomial reducible} to any
    language in \(P\) which is not \(\emptyset\) or \(\Sigma^*\).

    \begin{proof}
        Let there be two languages, \(A\) and \(B\).
    \end{proof}
\end{homeworkProblem}

\pagebreak

\begin{homeworkProblem}
    Let \(L = \{0^i 1^j : i > j\}\). Show that \(L \in TIME(n \lg n)\).

    \begin{proof}
        
    \end{proof}
\end{homeworkProblem}

\pagebreak

\begin{homeworkProblem}
     Prove that Graph isomorphism is in \(NP\). That is
     \[
         GI = \{\langle G, H \rangle : G, H \mbox{are isomorphic}\} \in NP
     \]

     Two graphs \(G = \langle V_G, E_G \rangle\) and \(H = \langle V_H, E_H \rangle\) are isomorphic
     iff there is a bijection \(f : V_G \rightarrow V_H\) such that \(\langle v, v' \rangle \in E_G\)
     if and only if \(\langle f(v), f(v') \rangle \in E_H\).
\end{homeworkProblem}

\pagebreak

\begin{homeworkProblem}
    Prove that Double Satisfaction Problem, defined as
    \[
        SAT2 = \{ \langle \phi \rangle \phi \mbox{is a 3NF-formula with at least two solutions}
    \]

    is \(NP\)-complete.

    \begin{proof}
        
    \end{proof}
\end{homeworkProblem}

\pagebreak

\begin{homeworkProblem}
    Prove that the class \(P\) is closed under union and complementation.
    \\

    \textbf{Part One} Prove that \(P\) is closed under union.

    \begin{proof}
        To prove that \(P\) is closed under union, we assume that there are two languages
        \(L_1\) and \(L_2\) with \(M_1\) and \(M_2\) as TMs that decide them.
        \\

        Assume that \(M_1\) runs in polynominal time, \(O(n^x)\) as well as
        \(M_2\), \(O(n^y)\).
        \\

        We will construct a new TM \(M\) that runs both \(M_1\) and \(M_2\) and show that it still
        runs in polynomial time.
        \\

        \(M = \)`` On input \(w\):
        \begin{enumerate}
            \item Run \(M_1\) on \(w\).
            \item Run \(M_2\) on \(w\).
            \item If one of the TMs accepted, accept, else reject.''
        \end{enumerate}

        The runtime of \(M\) can be determined as \(O(n^x) + O(n^y)\). Asymptotically,
        this is equal to the following: \(O(n^{z})\) where \(z = max(x, y)\).
        \\

        Thus we can see that P is closed under union.
    \end{proof}

    \textbf{Part Two} Prove that \(P\) is closed under complementation.

    \begin{proof}
        To prove that \(P\) is closed under complementation, we assume that there is a
        language \(L\) with a TM \(M\) that decides it.
        \\

        Assume that \(M\) runs in polynominal time, \(O(n^x)\).
        \\

        We will construct a new TM \(N\) that runs \(M\) and we will
        show that it still runs in polynomial time.
        \\

        \(N = \)`` On input \(w\):
        \begin{enumerate}
            \item Run \(M\) on \(w\).
            \item If \(M\) accepted, reject, else accept.''
        \end{enumerate}

        The construction of \(N\) is such that it decides the complement of the
        language that \(M\) decides. This TM also runs in \(O(n^x)\), which
        means it still runs in polynomial time.
        \\

        Thus we can see that P is closed under complementation.

    \end{proof}
\end{homeworkProblem}

\end{document}
