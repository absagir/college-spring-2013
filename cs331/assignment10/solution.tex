\documentclass{article}

\usepackage{fancyhdr}
\usepackage{lastpage}
\usepackage{extramarks}
\usepackage[usenames,dvipsnames]{color}
\usepackage{amsmath}
\usepackage{amsthm}
\usepackage{amsfonts}

\usepackage{tikz}

\topmargin=-0.45in
\evensidemargin=0in
\oddsidemargin=0in
\textwidth=6.5in
\textheight=9.0in
\headsep=0.25in

\linespread{1.1}

\pagestyle{fancy}
\lhead{\hmwkAuthorName}
\chead{\hmwkClass\ (\hmwkClassInstructor\ \hmwkClassTime): \hmwkTitle}
\rhead{\firstxmark}
\lfoot{\lastxmark}
\cfoot{}
\renewcommand\headrulewidth{0.4pt}
\renewcommand\footrulewidth{0.4pt}

\setlength\parindent{0pt}

\newcommand{\enterProblemHeader}[1]{
    \nobreak\extramarks{#1}{#1 continued on next page\ldots}\nobreak
    \nobreak\extramarks{#1 (continued)}{#1 continued on next page\ldots}\nobreak
}

\newcommand{\exitProblemHeader}[1]{
    \nobreak\extramarks{#1 (continued)}{#1 continued on next page\ldots}\nobreak
    \nobreak\extramarks{#1}{}\nobreak
}

\setcounter{secnumdepth}{0}
\newcounter{homeworkProblemCounter}

\newcommand{\homeworkProblemName}{}
\newenvironment{homeworkProblem}[1][Problem \arabic{homeworkProblemCounter}]{
    \stepcounter{homeworkProblemCounter}
    \renewcommand{\homeworkProblemName}{#1}
    \section{\homeworkProblemName}
    \enterProblemHeader{\homeworkProblemName}
}{
    \exitProblemHeader{\homeworkProblemName}
}

\newcommand{\problemAnswer}[1]{
\noindent\framebox[\columnwidth][c]{\begin{minipage}{0.98\columnwidth}#1\end{minipage}}
}

\newcommand{\homeworkSectionName}{}
\newenvironment{homeworkSection}[1]{
    \renewcommand{\homeworkSectionName}{#1}
    \subsection{\homeworkSectionName}
    \enterProblemHeader{\homeworkProblemName\ [\homeworkSectionName]}
}{
    \enterProblemHeader{\homeworkProblemName}
}

\newcommand{\hmwkTitle}{Homework\ \#10}
\newcommand{\hmwkDueDate}{April 25, 2013 at 11:59pm}
\newcommand{\hmwkClass}{CS331}
\newcommand{\hmwkClassTime}{9:00am}
\newcommand{\hmwkClassInstructor}{Professor Zhang}
\newcommand{\hmwkAuthorName}{Josh Davis}

\title{
    \vspace{2in}
    \textmd{\textbf{\hmwkClass:\ \hmwkTitle}}\\
    \normalsize\vspace{0.1in}\small{Due\ on\ \hmwkDueDate}\\
    \vspace{0.1in}\large{\textit{\hmwkClassInstructor\ \hmwkClassTime}}
    \vspace{3in}
}

\author{\textbf{\hmwkAuthorName}}
\date{}

\begin{document}

\maketitle

\pagebreak

\begin{homeworkProblem}
    Let \(G\) be a undirected and connected graph.
    \\

    Prove that \(EULERIAN-PATH = \{\langle G \rangle : G \mbox{ has an Eulerian
    path}\}\) is in \(P\).

    \begin{proof}
        To prove that \(EULERIAN-PATH\) is in \(P\), we will first prove that
        \(G\) has an Eulerian path iff every vertex in \(G\) has an even number
        degree. Then we will use this fact to create a TM to show that
        \(EULERIAN-PATH\) is in \(P\). We break up the first task into two
        parts.
        \\

        \textbf{Part One} If \(G\) has an \(EULERIAN-PATH\), then every vertex
        has an even degree.
        \\

        This is easy to prove, for every node \(v \in G\), in order to visit
        every edge only once, whenever we visit a node on an unused edge, there
        must be an accompanying unused edge to let us leave the node. This also
        applies to the very first/last node in the path. One edge must be used
        to leave it at the start of the path and one edge must be used to exit
        onto it at the end of the path.
        \\

        Thus if a graph \(G\) has an \(EULERIAN-PATH\), then every vertex has
        an even degree because we need to enter it by an unused edge and leave
        it by an unused path.
        \\

        \textbf{Part Two} If every vertex has an even degree, then \(G\) has an
        \(EULERIAN-PATH\).
        \\

        To show this, we will use a proof by induction.
        \\

        \textbf{Base Case} We let the number of vertices \(\left| V \right| =
        1\). We can see that if there is one vertex, then the number of
        \(\left| E \right| = 0\) which is even. Thus with one vertex and 0
        number of vertices, the Euler path starts and ends on the same vertex.
        \\

        \textbf{Induction Step} Next we will try to show that if we have a graph \(G\)
        with vertices \(\left| V \right| = n\), that adding another vertex still results
        in an Eulerian path.
        \\

        Let \(G\) be a graph with \(\left V \right| = n\) vertices and each
        vertice has an even degree. We can assume the induction hypothesis in
        that there there is an Eulerian path for \(G\).
        \\

        Let \(s \in V\) be the starting and ending vertex. If we take a new
        vertex, say \(t\) we can add \(t\) to our Eulerian path. To do so, we
        can add \(t\) at the end of the path so that \(t\) has a degree of two.
        The first edge would lead from the vertex that connected to \(s\) and
        then the second edge would end on \(s\). Thus the \(\left| V \right| =
        n + 1\) and the degree of \(t\) is even.


        Now we will show that \(EULERIAN-PATH\) is in \(P\) by constructing a
        TM \(M\) that runs in polynomial time.
        \\

        Since we have constructed a TM that runs in polynomial time,
        \(EULERIAN-PATH\) is in \(P\) and our proof is complete.
    \end{proof}
\end{homeworkProblem}

\pagebreak

\begin{homeworkProblem}
    Let \(TSP = \{ \langle G, K \rangle : G \mbox{ has a Hamiltonian path of
    length less than } k\}\).
    \\

    Prove that \(TSP\) is \(NP\)-complete.

    \begin{proof}
        To prove that \(TSP\) is \(NP\)-complete, we will show that \(HAMPATH
        \leq_p TSP\). This will show that \(TSP\) is \(NP-complete\) based on
        \textbf{Theorem 7.36}.
        \\

        First we will show that \(TPS \in NP\). This is easy because we can
        easily construct a verifier that given a certificate \(c\), we just
        start at the first node in \(c\) and iterate over it, adding up the
        lengths until there are no more nodes in \(c\). If this is less than
        \(k\), then we accept, else we reject. Thus since this is polynomially
        verifiable, \(TSP \in NP\).
        \\

        Now to prove that \(TSP\) is \(NP\)-complete, we need to show that
        a problem that is \(NP\)-complete can be polynomially reduced to it.
        We will now show that \(HAMPATH \leq_p TSP\).
        \\

        Since \(HAMPATH \in NP\), there exists a NTM \(M\) that can solve it
        nondeterministically.

    \end{proof}
\end{homeworkProblem}

\pagebreak

\begin{homeworkProblem}
    Prove that a language is in \(co-NP\) iff it has a polynomial time disqualifier.

    \begin{proof}
        
    \end{proof}
\end{homeworkProblem}

\pagebreak

\begin{homeworkProblem}
    Prove that a language is in \(co-NP\) iff it has a polynomial time disqualifier.

    \begin{proof}
        To prove that a language in \(coNP\) iff it has a polynomial time disqualifier, we
        will break it up and prove it in two parts.
        \\

        \textbf{Part One} If a language \(L \in coNP\), then it has a polynomial time disqualifier.
        \\

        \begin{proof}
            
        \end{proof}

        \textbf{Part Two} If a language \(L\) has a polynomial time disqualifier, then \(L \in coNP\).
        \\

        \begin{proof}
            
        \end{proof}
    \end{proof}
\end{homeworkProblem}

\pagebreak

\begin{homeworkProblem}
    \textbf{Part One} Prove that \(coNP\) is closed under polynomial-time
    reductions; that is, if \(L_1 \leq_p L_2\) and \(L_2 \in coNP\), then
    \(L_1 \in coNP\).

    \begin{proof}
        
    \end{proof}

    \textbf{Part Two} Prove that if a \(coNP\)-complete language \(L\) is
    in \(NP\) then \(coNP = NP\).

    \begin{proof}
        
    \end{proof}
\end{homeworkProblem}

\pagebreak

\begin{homeworkProblem}
    \textbf{Part One} For any language class \(C\), \(P^C\) is closed under
    complementation; that is, \(L \in P^C \iff \overline{L} \in P^C\).

    \begin{proof}
        
    \end{proof}

    \textbf{Part Two} For any language class \(C\) that is closed under
    complementation, \(C \subset coNP \iff C \subset NP\).

    \begin{proof}
        
    \end{proof}

    \textbf{Part Three} Prove \(P^P \subset P\).

    \begin{proof}
        
    \end{proof}

    \textbf{Part Four} Prove that \(NP^P \subset NP\).

    \begin{proof}
        
    \end{proof}
\end{homeworkProblem}

\end{document}
