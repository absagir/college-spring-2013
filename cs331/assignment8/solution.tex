\documentclass{article}

\usepackage{fancyhdr}
\usepackage{lastpage}
\usepackage{extramarks}
\usepackage[usenames,dvipsnames]{color}
\usepackage{amsmath}
\usepackage{amsthm}
\usepackage{amsfonts}

\usepackage{tikz}

\topmargin=-0.45in
\evensidemargin=0in
\oddsidemargin=0in
\textwidth=6.5in
\textheight=9.0in
\headsep=0.25in

\linespread{1.1}

\pagestyle{fancy}
\lhead{\hmwkAuthorName}
\chead{\hmwkClass\ (\hmwkClassInstructor\ \hmwkClassTime): \hmwkTitle}
\rhead{\firstxmark}
\lfoot{\lastxmark}
\cfoot{}
\renewcommand\headrulewidth{0.4pt}
\renewcommand\footrulewidth{0.4pt}

\setlength\parindent{0pt}

\newcommand{\enterProblemHeader}[1]{
    \nobreak\extramarks{#1}{#1 continued on next page\ldots}\nobreak
    \nobreak\extramarks{#1 (continued)}{#1 continued on next page\ldots}\nobreak
}

\newcommand{\exitProblemHeader}[1]{
    \nobreak\extramarks{#1 (continued)}{#1 continued on next page\ldots}\nobreak
    \nobreak\extramarks{#1}{}\nobreak
}

\setcounter{secnumdepth}{0}
\newcounter{homeworkProblemCounter}

\newcommand{\homeworkProblemName}{}
\newenvironment{homeworkProblem}[1][Problem \arabic{homeworkProblemCounter}]{
    \stepcounter{homeworkProblemCounter}
    \renewcommand{\homeworkProblemName}{#1}
    \section{\homeworkProblemName}
    \enterProblemHeader{\homeworkProblemName}
}{
    \exitProblemHeader{\homeworkProblemName}
}

\newcommand{\problemAnswer}[1]{
\noindent\framebox[\columnwidth][c]{\begin{minipage}{0.98\columnwidth}#1\end{minipage}}
}

\newcommand{\homeworkSectionName}{}
\newenvironment{homeworkSection}[1]{
    \renewcommand{\homeworkSectionName}{#1}
    \subsection{\homeworkSectionName}
    \enterProblemHeader{\homeworkProblemName\ [\homeworkSectionName]}
}{
    \enterProblemHeader{\homeworkProblemName}
}

\newcommand{\hmwkTitle}{Homework\ \#8}
\newcommand{\hmwkDueDate}{April 11, 2013 at 11:59pm}
\newcommand{\hmwkClass}{CS331}
\newcommand{\hmwkClassTime}{9:00am}
\newcommand{\hmwkClassInstructor}{Professor Zhang}
\newcommand{\hmwkAuthorName}{Josh Davis}

\title{
    \vspace{2in}
    \textmd{\textbf{\hmwkClass:\ \hmwkTitle}}\\
    \normalsize\vspace{0.1in}\small{Due\ on\ \hmwkDueDate}\\
    \vspace{0.1in}\large{\textit{\hmwkClassInstructor\ \hmwkClassTime}}
    \vspace{3in}
}

\author{\textbf{\hmwkAuthorName}}
\date{}

\begin{document}

\maketitle

\pagebreak

\begin{homeworkProblem}
    Let \(A \leq_{L} B\) mean that \(A \leq_{T} B\) with additional condition
    that the oracle Turing machine \(M^B\) that solves \(A\) queries the oracle
    for \(B\) only once, at the very last step.

    \begin{proof}

    \end{proof}
\end{homeworkProblem}

\pagebreak

\begin{homeworkProblem}
    Describe two different Turing machines, \(M_1\) and \(M_2\) such that when
    started on any input, \(M_1\) outputs \(\langle M_2 \rangle\) and \(M_2\)
    outputs \(\langle M_1 \rangle\).
    \\

    The two TMs are quite similar to the SELF program in that the first TM will
    print the encoding of the second machine and leave it on the tape. The second TM will
    then use that to compute what the first one is. We can define the two TM's like so:
    \\

    \(M_1 = P_{\langle M_2 \rangle}\)
    \\

    \(M_2 = \)`` On input \(\langle M \rangle\) where M is a TM:
    \begin{enumerate}
        \item Compute \(q(\langle M \rangle)\).
        \item Print newly computed TM and halt.''
    \end{enumerate}

    Similar to how Sipser explains the behavior for \(SELF\), we'll explain the
    behavior for this construction.

    \begin{enumerate}
        \item First \(M_1\) runs. It prints \(\langle M_2 \rangle\) on the tape.
        \item \(M_2\) starts. It looks at the tape and finds its own input, \(\langle M_2 \rangle\).
        \item \(M_2\) uses the lemma 6.1 in the book to calculate \(q(\langle M_2 \rangle)\) which is equal to \(\langle M_1 \rangle\).
        \item \(M_2\) then prints this newly computed description and halts.
    \end{enumerate}

    \textbf{Note:} This whole machine has the description of \(M_1 M_2\). This
    works because a program is just a string and thus two can be concatenated.
\end{homeworkProblem}

\pagebreak

\begin{homeworkProblem}

    \begin{proof}

    \end{proof}
\end{homeworkProblem}

\pagebreak

\begin{homeworkProblem}
    Let \(SELF_{TM} = \{ \langle M \rangle : L(M) = \{\langle M \rangle \}\}\).
    Prove that neither \(SELF_{TM}\) nor \(\overline{SELF_{TM}}\) is
    Turing-recognizable.

    \begin{proof}

    \end{proof}
\end{homeworkProblem}

\pagebreak

\begin{homeworkProblem}
    Prove that the class \(P\) is closed under union and complementation.
    \\

    \textbf{Part One} Prove that \(P\) is closed under union.

    \begin{proof}

    \end{proof}

    \textbf{Part Two} Prove that \(P\) is closed under complementation.

    \begin{proof}

    \end{proof}
\end{homeworkProblem}

\end{document}
